\documentclass{article}
\usepackage[utf8]{inputenc}

\title{Writing 1}
\author{Li Miao   miao0044}
\date{02/05/2021}

\usepackage{natbib}
\usepackage{graphicx}

\begin{document}

\maketitle
\noindent
BBefore discussing about the importance of ethics in AI, we must clarify one question first: “Why do human society has ethics?”
\newline
\newline
We all know that, only one kind of gene can be successful, which is the gene who benefits themselves, thus increases its own chance of survival. From the author’s explanation, we can see that in our society with ethics, people are encouraged to do things to benefit others and the society. Why did the individuals in our society with altruism qualities successfully survived till today, but not the selfish genes? The answer is, there is no truly altruistic individual. The only reason why ethics and altruism genes successfully passed down to today is: While having ethics benefits the society, it also benefits the individual itself.
\newline
\newline
Now it becomes clear that is ethics important in AI or not. The answer depends on only one question: In the future society, would AI with ethics more fit to survive than AI without ethics? The answer could be yes or no. Nobody today knows how the future word looks like, and what qualities are needed for an AI to win the competition against other AIs.
\newline
\newline
I disagree with the author’s opinion and intention. The author is looking from the human’s perspective to comment and judge AI. But things more likely to happen is: Human would keep losing in the competition against AI. When preforming a task, humans has extremely low efficiency: they get tired easily; making minor mistake all the times; needs to eat, drink, go to restroom, go sleep… At same time, AI could do their job more flawlessly, accurately, and without the need to rest and sleep, and only need some energy to “eat”. Eventually, AI would win the competition against human being for more and more tasks, and finally replace human.
\newline
\newline
Therefore, discussing about ethics in AI from human’s perspective is very naïve and meaningless. Just like no matter how hard we try; we cannot explain calculus and theory quantum physics to a cat. By time, AI will develop their own ethics system to fit their society in future, which would be way more advanced, complicated, and far beyond our comprehension.
\newline
\newline
About the third example, Sharing the Wealth: This is a very old topic, it is as old as the entire human history. “The rich get richer and the poor get poorer.” This golden rule keeps repeating itself centuries after centuries, in both western cultures and eastern cultures. The difference between the rich and the poor keeps increasing. When it finally exceeds curtain ratio, the whole system collapses; and the poor would start a revolution to defeat the rich, then re-distribute the wealth. After that, some of the poor become the new rich, and the whole cycle repeats itself again.
\newline
\newline
However, there are two ways how AI could change this game. First, the rich would no longer need the poor. Unlike the old time, the slave-masters require slaves to work for them. As AI becoming more and more advanced, there are more and more jobs which could be replaced by AI. Today’s rich can rely on AI to do the work, the human labors are becoming less and less needed. Second, the rich would no longer fear the poor. In cold weapon era, the population of an army meant a lot. It was impossible for 100 well-trained soldiers to defeat 1,000,000 normal peasants. But now, with bio-chem weapon, nuclear weapon, and other advanced technologies, even 1 person can easily defeat a 1,000,000-man army. As technology improves, the revolution from the poor becomes more and more difficult to execute.
\newline
\newline
Thus, I disagree the future of social norm mentioned in the author’s paper. Since the rich would no longer need and fear the poor (which had never happened in history), the social norm would change to a state which we never seen before. It is very likely that the wealth difference would keep increasing, until one day the poor are no longer able to survive anymore.
\newline
\newline
Overall, I think the author’s opinion is based on the very beginning of AI’s evolution. I believe, in the not too distance future, AI would have brand new ideas about ethics, which are smarter and better than our ideas. So, instead of discussing the ethics system for AI, we should leave this topic for AI to study and answer.

\newline

\end{document}
